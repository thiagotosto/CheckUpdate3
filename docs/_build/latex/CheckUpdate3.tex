%% Generated by Sphinx.
\def\sphinxdocclass{report}
\documentclass[letterpaper,10pt,openany,oneside,portuges]{sphinxmanual}
\ifdefined\pdfpxdimen
   \let\sphinxpxdimen\pdfpxdimen\else\newdimen\sphinxpxdimen
\fi \sphinxpxdimen=.75bp\relax

\usepackage[utf8]{inputenc}
\ifdefined\DeclareUnicodeCharacter
  \DeclareUnicodeCharacter{00A0}{\nobreakspace}
\fi
\usepackage{cmap}
\usepackage[T1]{fontenc}
\usepackage{amsmath,amssymb,amstext}
\usepackage[portuguese]{babel}
\usepackage{times}
\usepackage[Sonny]{fncychap}
\usepackage{longtable}
\usepackage{sphinx}

\usepackage{geometry}
\usepackage{multirow}
\usepackage{eqparbox}

% Include hyperref last.
\usepackage{hyperref}
% Fix anchor placement for figures with captions.
\usepackage{hypcap}% it must be loaded after hyperref.
% Set up styles of URL: it should be placed after hyperref.
\urlstyle{same}

\addto\captionsportuges{\renewcommand{\figurename}{Fig.}}
\addto\captionsportuges{\renewcommand{\tablename}{Table}}
\addto\captionsportuges{\renewcommand{\literalblockname}{Listing}}

\addto\extrasportuges{\def\pageautorefname{page}}

\setcounter{tocdepth}{1}



\title{CheckUpdate3 Documentation}
\date{abr 03, 2017}
\release{}
\author{Thiago Tosto}
\newcommand{\sphinxlogo}{}
\renewcommand{\releasename}{Versão}
\makeindex

\begin{document}
\if\catcode`\"\active\shorthandoff{"}\fi
\maketitle
\sphinxtableofcontents
\phantomsection\label{\detokenize{index::doc}}



\chapter{Introdução}
\label{\detokenize{intro:introducao}}\label{\detokenize{intro:welcome-to-checkupdate3-s-documentation}}\label{\detokenize{intro::doc}}
Aplicação de gerenciamento de inventário físico. Vempara substituir planilhas que fazem esse controle.


\chapter{Tecnologias utilizadas}
\label{\detokenize{tecnologias::doc}}\label{\detokenize{tecnologias:tecnologias-utilizadas}}\begin{itemize}
\item {} 
Python

\item {} 
SQAlchemy

\item {} 
MySql(MariaDB)

\item {} 
Flask

\end{itemize}


\chapter{Módulos}
\label{\detokenize{classes:modulos}}\label{\detokenize{classes::doc}}
inconsistent\_check: checa inconsistencia antes de persistencia no banco.
dbconnect: classe de conexão no banco.
form2db: tradutor do formulário para o banco.
app.py: criadores de rotas do flask.


\section{App.py}
\label{\detokenize{classes:app-py}}
rotas:
\begin{itemize}
\item {} 
index: homepage, template=index.html.

\item {} 
consulta: resultado da consulta, template=consulta.html

\item {} 
adiciona:

\end{itemize}


\chapter{Funcionalidades}
\label{\detokenize{funcionalidades:funcionalidades}}\label{\detokenize{funcionalidades::doc}}\begin{itemize}
\item {} 
Cadastro de servidor

\item {} 
Edição de servidor

\item {} 
Busca de servidor

\item {} 
Importação de Planilha

\item {} 
Exportação para Planilha

\end{itemize}


\chapter{Fluxo}
\label{\detokenize{fluxo::doc}}\label{\detokenize{fluxo:fluxo}}\begin{itemize}
\item {} \begin{description}
\item[{Página home:}] \leavevmode\begin{itemize}
\item {} 
Consulta -\textgreater{} Query no banco -\textgreater{} apresenta consulta front end (extende: Editar)

\item {} 
Editar -\textgreater{} Carrega as mudanças -\textgreater{} Submit mudanças -\textgreater{} Checa inconsistencias -\textgreater{} Persiste no banco.

\item {} 
Adciona -\textgreater{} Carrega valores -\textgreater{} Submit mudanças -\textgreater{} Checa inconsistências -\textgreater{} Periste no banco.

\item {} 
\sphinxstyleemphasis{Importar -\textgreater{} Versão 2.0}

\item {} 
\sphinxstyleemphasis{Exportar -\textgreater{} Versão 2.0}

\end{itemize}

\end{description}

\end{itemize}



\renewcommand{\indexname}{Índice}
\printindex
\end{document}